\documentclass[12p]{article}
\usepackage{amsmath, amssymb, amsthm}
\usepackage[dvipsname]{xcolor}
\usepackage{framed}
\usepackage[top=1cm, bottom=1cm, left=3cm, right=3cm]{geometry}
\usepackage{tcolorbox}
\usepackage{etoolbox}
\usepackage{tikz}

 

% Set page background to black and text to white
%\pagecolor{black}
%\color{white}

% Define counters for environments
\newcounter{usercounter}
\newcounter{answercounter}
\newcounter{reasoningcounter}

% Define the user environment
\newenvironment{user}[1][]
{\refstepcounter{usercounter}%
\begin{tcolorbox}[
    title=User \theusercounter: #1,
    colback=white, %black,
    colframe=gray,
    coltext=black, %white,
    fonttitle=\bfseries\large,
    top=2mm,
    bottom=2mm,
    enlarge top initially by=5mm
]
\itshape
}
{\end{tcolorbox}}

% Define the reasoning environment
\newenvironment{reasoning}[1][]
{\refstepcounter{reasoningcounter}%
\begin{tcolorbox}[
    title=Reasoning \thereasoningcounter: #1,
    colback=white, %black,
    colframe=gray,
    coltext=black, %white,
    fonttitle=\bfseries\normalsize,
    top=2mm,
    bottom=2mm,
    enlarge top initially by=5mm
]
\small
}
{\end{tcolorbox}}

\newenvironment{answer}[1][]
{\refstepcounter{answercounter}%
\begin{tcolorbox}[
    title=Answer \theanswercounter: #1,
    colback=white, %black,
    colframe=gray,
    coltext=black, %white,
    fonttitle=\bfseries\normalsize,
    top=2mm,
    bottom=2mm,
    enlarge top initially by=5mm
]
\small
}
{\end{tcolorbox}}


\title{LatexColab Template}
\author{}
\date{March 2025}

\begin{document}

\maketitle

% example usage

\begin{user}
YOUR QUERY GOES HERE. THE MODEL AND STATUS GOES BELOW. SPECIFY status=start TO KICKSTART THE REASONER.
THE status WILL TURN TO ID_TIMESTAMP ONCE THE ANSWER environment IS FINALIZED. 

%parameters: model=o1, status=start
\end{user}



\begin{user}
Create the Caley graph of F4.
%parameters: model=o1, status=completed_20250324_181746_815513
\end{user}


\begin{answer}[by openai/o1 (generated in 0 minutes and 46 seconds.)]
\[
\textbf{Description of the Cayley Graph of }F_4\textbf{:}
\]
The free group on four generators, often denoted 
\(\displaystyle F_4 = \langle a,b,c,d\rangle,\)
is the group of all finite “words” in the symbols \(a,a^{-1},b,b^{-1},c,c^{-1},d,d^{-1}\) with the only relation that each generator is canceled by its inverse (e.g.\ \(a a^{-1} = e\), \(b b^{-1} = e\), etc.).

Its Cayley graph (with respect to these four generators) is an infinite, regular tree of degree \(8\), where each vertex corresponds to a distinct group element (word), and from each vertex there are eight edges labeled by \(a,a^{-1},b,b^{-1},c,c^{-1},d,d^{-1}\). Traveling along an edge labeled by a generator \(x\in\{a,b,c,d\}\) corresponds to right-multiplying the current word by \(x\). Traveling back along that same edge corresponds to right-multiplying by \(x^{-1}\). Because \(F_4\) has no nontrivial relations, no nontrivial loops can form in this graph, yielding an infinite tree.

Below is a small portion of this infinite graph, just showing the identity element \(e\) (the empty word) and the first “ring” of neighbors labeled \(\{a,b,c,d\}\). Each of those neighbors connects back to \(e\) with the inverse of its label. Beyond this layer, each of those vertices again branches out in six more directions (since one direction leads back to \(e\)), continuing ad infinitum.

\[
\begin{tikzpicture}[%
   auto,
   node distance=2.3cm,
   >=stealth,
   every node/.style={draw,circle},
   every loop/.style={looseness=7}
]
   % Identity vertex
   \node (e) {\(e\)};

   % First ring: a, b, c, d around e
   \node (a) [above left of=e] {\(a\)};
   \node (b) [above right of=e] {\(b\)};
   \node (c) [below right of=e] {\(c\)};
   \node (d) [below left of=e] {\(d\)};

   % Draw edges from e to a,b,c,d, labeling them
   \path (e) edge node [swap,midway] {\(a\)} (a);
   \path (e) edge node [midway] {\(b\)} (b);
   \path (e) edge node [midway] {\(c\)} (c);
   \path (e) edge node [swap,midway] {\(d\)} (d);

   % In the Cayley graph, the same edge in the opposite direction is labeled by the inverse
   % You can depict this by a single undirected edge with a double label, or by separate arrows.
   % Here, we show separate arrows for clarity:

   \path (a) edge [bend left=20] node {\(a^{-1}\)} (e);
   \path (b) edge [bend left=20] node {\(b^{-1}\)} (e);
   \path (c) edge [bend left=20] node {\(c^{-1}\)} (e);
   \path (d) edge [bend left=20] node {\(d^{-1}\)} (e);

   % Optional: you could continue expanding from each of a,b,c,d to illustrate even more vertices.
\end{tikzpicture}
\]

\[
\text{This diagram shows only a local portion. The full Cayley graph is an infinite 8-regular tree.}
\]
\end{answer}